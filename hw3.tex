\documentclass[submit]{harvardml}

% FDV: Check all frontmatter for years, due dates, and references for
% sections of the textbook, etc.
\course{CS181-S22}
\assignment{Assignment \#3}
\duedate{11:59pm EST, March 11, 2022}

\usepackage[OT1]{fontenc}
\usepackage[colorlinks,citecolor=blue,urlcolor=blue]{hyperref}
\usepackage[pdftex]{graphicx}
\usepackage{subfig}
\usepackage{fullpage}
\usepackage{amsmath}
\usepackage{amssymb}
\usepackage{color}
\usepackage{soul}
\usepackage{todonotes}
\usepackage{listings}
\usepackage{common}
\usepackage{enumitem}
\usepackage{bm}
\newcommand{\B}{\text{B}}
\newcommand{\Beta}{\text{Beta}}
\usepackage{pythonhighlight}
\usepackage[mmddyyyy,hhmmss]{datetime}


\setlength\parindent{0em}

\definecolor{verbgray}{gray}{0.9}

\lstnewenvironment{csv}{%
  \lstset{backgroundcolor=\color{verbgray},
  frame=single,
  framerule=0pt,
  basicstyle=\ttfamily,
  columns=fullflexible}}{}
  
\usepackage{amsmath}
\DeclareMathOperator*{\argmax}{arg\,max}
\DeclareMathOperator*{\argmin}{arg\,min}

\begin{document}

\begin{center}
{\Large Homework 3: Bayesian Methods and Neural Networks}\\
\end{center}

% FDV: Update for resources, accuracy of submit information **especially any colab components**
% FDV: Make sure to emphasize any plots must be in the pdf submission, we will not be checking code / source 
\subsection*{Introduction}

This homework is about Bayesian methods and Neural Networks.  Section 2.9 in the textbook as well as reviewing MLE and MAP will be useful for Q1. Chapter 4 in the textbook will be useful for Q2.

Please type your solutions after the corresponding problems using this
\LaTeX\ template, and start each problem on a new page.

Please submit the \textbf{writeup PDF to the Gradescope assignment `HW3'}. Remember to assign pages for each question.  \textbf{All plots you submit must be included in your writeup PDF.  }We will not be checking your code / source files except in special circumstances. 

Please submit your \textbf{\LaTeX file and code files to the Gradescope assignment `HW3 - Supplemental'}. 

% FDV: Last year, we pointed folks to
% https://www.cs.ubc.ca/~murphyk/Papers/bayesGauss.pdf
% and folks ended up basically copying from it and not doing any derivation
% For this year, I modified to ask folks to do the full derivation in
% 1.1 but then give them the formula for the marginal likelihood so
% they don't have to derive that.  Open to other variations:
% basically, I think it's probably okay for one part to have a longer
% derivation, but if folks think it's too much, we could have students
% refer to the paper above again or just give them the formula for 1.1
% and ask them to simply interpret it.

\newpage


\begin{problem}[Bayesian Methods]

  This question helps to build your understanding of making
  predictions with a maximum-likelihood estimation (MLE), a maximum a
  posterior estimator (MAP), and a full posterior predictive.

  Consider a one-dimensional random variable $x = \mu + \epsilon$,
  where it is known that $\epsilon \sim N(0,\sigma^2)$.  Suppose we
  have a prior $\mu \sim N(0,\tau^2)$ on the mean. You observe iid data $\{x_i\}_{i=1}^n$ (denote the data as $D$).


\textbf{We derive the distribution of $x|D$ for you.}

\textbf{The full posterior predictive is computed using:}

$$p(x|D)=\int p(x, \mu|D) d\mu =\int p(x|\mu)p(\mu|D) d\mu $$


\textbf{One can show that, in this case, the full posterior predictive distribution has a nice analytic
  form:}
   \begin{align}
        x|D \sim \mathcal{N}\Bigg(\frac{\sum_{x_i\in D}x_i}{n+\frac{\sigma^2}{\tau^2}}, (\frac{n}{\sigma^2}+\frac{1}{\tau^2})^{-1} + \sigma^2\Bigg)
     \end{align}

\begin{enumerate}

  \item Derive the distribution of $\mu|D$.

  \item 
  In many problems, it is often difficult to
  calculate the full posterior because we need to marginalize out the parameters as above (here,
  the parameter is $\mu$). We can mitigate this problem by plugging in
  a point estimate of $\mu^*$ rather than a distribution.

  a) Derive the MLE
  estimate $\mu_{MLE}$.
  % using $p(D|\mu)$.
  
  b) Derive the MAP estimate $\mu_{MAP}$. % using $p(\mu|D)$

  c) What is the relation between $\mu_{MAP}$ and the mean of $x|D$?

  d) For a fixed value of $\mu=\mu^*$, what is the distribution of $x|\mu^*$? Thus, what is the distribution of $x|\mu_{MLE}$ and $x|\mu_{MAP}$?

  e) Is the variance of $x|D$ greater or smaller than the variance of $x|\mu_{MLE}$? What is the limit of the variance of $x|D$ as $n$ tends to infinity? Explain why this is intuitive.


\item Let us compare $\mu_{MLE}$  and $\mu_{MAP}$. There are three cases to consider:

 a) Assume $\sum_{x_i \in D} x_i=0$. What are the values of $\mu_{MLE}$ and $\mu_{MAP}$?

 b) Assume $\sum_{x_i \in D} x_i>0$. Is $\mu_{MLE}$ greater than $\mu_{MAP}$?

 c) Assume $\sum_{x_i \in D} x_i<0$. Is $\mu_{MLE}$ greater than $\mu_{MAP}$?
  
    
\item Compute:

$$\lim_{n \rightarrow \infty} \frac{\mu_{MAP}}{\mu_{MLE}}$$

  \end{enumerate}

  \end{problem}

\subsection*{Solution:}
1.1 We use Bayes' to obtain:
\begin{align*}
    p(\mu|D) &\propto p(D|\mu, \tau) p(\mu)\\
    \intertext{Next, we use the Normal PDF for both terms to get:}
    &= \exp(-\frac{1}{2\sigma^2} \sum_{i=1}^n (x_i - \mu)^2 \exp(-\frac{1}{2\tau^2} *\mu^2))\\
    \intertext{Expansion and simplification yield:}
    &= \exp(-\frac{1}{2}(\mu^2*(\frac{1}{\tau^2}+\frac{n}{\sigma^2}) - 2\frac{\mu}{\sigma^2}(\sum x_i)))\\
    \intertext{Further simplification and pattern matching yields that the distribution is as follows:}
    \mu | D &\sim \mathcal{N} (\frac{\sum x_i}{\frac{\sigma^2}{\tau^2} + n}, \big[\frac{1}{\tau^2} + \frac{n}{\sigma^2}\big]^{-1})
\end{align*}

1.2 (a)  To find the MLE, we first find the likelihood function. 
\begin{align*}
    L(D; \mu) &= \Pi_{i=1}^n p(x_i|\mu)\\
    &= \Pi_{i=1}^n \mathcal{N}(\mu, \sigma^2)\\
    \intertext{Taking the log yields:}
    l(D; \mu) &= \sum_{i=1}^n \log(\mathcal{N}(\mu, \sigma^2))\\
    &= -\sum_{i=1}^n \frac{(x_i-\mu)^2}{\sigma^2}
    \intertext{Next we take the derivative and set it equal to 0.}
    0 &= \sum_{i=1}^n \frac{2(x_i - \mu)}{\sigma^2}\\
    0 &= \sum_{i=1}^n x_i - \mu\\
    n\mu &= \sum_{i=1}^n x_i\\
    \hat{\mu}_{MLE} &= \frac{\sum_{i=1}^n x_i}{n}
\end{align*}

1.2 (b) The MAP estimate is the $\mu$ that maximizes $p(\mu|D)$. Since $\mu|D$ is distributed Normally, and Normal distributions are symmetric about their mean, $\mu$ that maximizes $p(\mu|D)$ is the mean of the distribution. Therefore, $\hat{\mu}_{MAP} = \frac{\sum_{i=1}^n x_i}{n + \frac{\sigma^2}{\tau^2}}$.

1.2 (c) $\mu_{MAP}$ and the mean of $x|D$ are the same. This is because $\mu_{MAP}$ maximizes the posterior probability on the distribution, and $x|D$ is Normally distributed, and the mean of Normal distributions is the point that maximizes this probability.

1.2 (d) For a fixed value of $\mu = \mu^*$, the distribution is 
\begin{center}
    $x|\mu^* \sim \mathcal{N}(\mu^*, \sigma^2)$
\end{center}
Therefore, the distributions of $x|\mu_{MLE}$ and $x|\mu_{MAP}$ are 
\begin{center}
    $x|\mu_{MLE} \sim \mathcal{N}(\mu_{MLE}, \sigma^2)$
    $x|\mu_{MAP} \sim \mathcal{N}(\mu_{MAP}, \sigma^2)$
\end{center}

1.2 (e) The variance of $x|D$ is greater than the variance of $x|\mu_{MLE}$ because the variance of $x|D$ includes an additional $\Big[ \frac{n}{\sigma^2} + \frac{1}{\tau^2} \Big]^{-1}$ term.
The limit of the variance of $x|D$ as $n$ tends to infinity is equal to the variance of of $x|\mu_{MLE}$, since this additional terms scales inversely with $n$. This makes intuitive sense because as the dataset size approaches infinity, the effect of the prior on the likelihood function becomes smaller and smaller.

1.3 (a) Plugging in to our results from 1.2(a) and 1.2(b), we have that $\mu_{MLE} = 0$ and $\mu_{MAP} = 0$.

1.3 (b) We have that $\mu_{MLE} = \frac{\sum_{i=1}^n x_i}{n}$ and $\mu_{MAP} = \frac{\sum_{i=1}^n x_i}{n+\frac{\sigma^2}{\tau^2}}$. Therefore, if $\sum_{i=1}^n x_i > 0$, then $\mu_{MLE}$ is greater than $\mu_{MAP}$.

1.3 (c) Using the same reasoning as in part (b), $\sum_{i=1}^n x_i < 0$ means that $\mu_{MLE}$ is not greater than $\mu_{MAP}$.

1.4 $$
    \lim_{n \rightarrow \infty} \frac{\mu_{MAP}}{\mu_{MLE}} = \lim_{n \rightarrow \infty} \frac{n}{n + \frac{\sigma^2}{\tau^2}} = 1
$$
\newpage

\begin{problem}[Bayesian Frequentist Reconciliation]
    In this question, we connect the Bayesian version of regression with the frequentist view we have seen in the first week of class by showing how appropriate priors could correspond to regularization penalities in the frequentist world, and how the models can be different.
    
    Suppose we have a $(p+1)$-dimensional labelled dataset $\mathcal{D} = \{(\mathbf{x}_i, y_i)\}_{i=1}^N$. We can assume that $y_i$ is generated by the following random process: $$y_i = \mathbf{w}^\top\mathbf{x}_i + \epsilon_i$$ where all $\epsilon_i \sim \mathcal{N}(0,\sigma^2)$ are iid. Using matrix notation, we denote
    \begin{align*}
      \mathbf{X} &= \begin{bmatrix}\mathbf{x}_1 & \ldots & \mathbf{x}_N\end{bmatrix}^\top \in \mathbb{R}^{N \times p} \\
      \mathbf{y} &= \begin{bmatrix} y_1 & \dots & y_N \end{bmatrix}^\top  \in \mathbb{R}^N \\
      \mathbf{\epsilon} &= \begin{bmatrix} \epsilon_1 & \dots & \epsilon_N \end{bmatrix}^\top \in \mathbb{R}^N.
    \end{align*}
    
    Then we can write have $\mathbf{y} = \mathbf{X}\mathbf{w} + \mathbf{\epsilon}$. Now, we will suppose that $\mathbf{w}$ is random as well as our labels! We choose to impose the Laplacian prior $p(\mathbf{w})=\frac{1}{2\tau}\exp\left(-\frac{\|\mathbf{w}-\mathbf{\mu}\|_1}{\tau}\right)$, where $\|\mathbf{w}\|_1=\sum_{i=1}^p |w_i|$ denotes the $L^1$ norm of $\mathbf{w}$, $\mathbf{\mu}$ the location parameter, and $\tau$ is the scale factor.
    
    \begin{enumerate}
    
        \item Compute the posterior distribution $p(\mathbf{w}|\mathbf{X}, \mathbf{y})$ of $\mathbf{w}$ given the observed data $\mathbf{X}, \mathbf{y}$, up to a normalizing constant. You \textbf{do not} need to simplify the posterior to match a known distribution.
        
        \item Determine the MAP estimate $\mathbf{w}_{\mathrm{MAP}}$ of $\mathbf{w}$. You may leave the answer as the solution to an equation. How does this relate to regularization in the frequentist perspective? How does the scale factor $\tau$ relate to the corresponding regularization parameter $\lambda$? Provide intuition on the connection to regularization, using the prior imposed on $\mathbf{w}$.
        
        \item Based on the previous question, how might we incorporate prior expert knowledge we may have for the problem? For instance, suppose we knew beforehand that $\mathbf{w}$ should be close to some vector $\mathbf{v}$ in value. How might we incorporate this in the model, and explain why this makes sense in both the Bayesian and frequentist viewpoints.
        
        \item As $\tau$ decreases, what happens to the entries of the estimate $\mathbf{w}_{\mathrm{MAP}}$? What happens in the limit as $\tau \to 0$?
        
        \item Consider the point estimate $\mathbf{w}_{\mathrm{mean}}$, the mean of the posterior $\mathbf{w}|\mathbf{X},\mathbf{y}$. Further, assume that the model assumptions are correct. That is, $\mathbf{w}$ is indeed sampled from the posterior provided in subproblem 1, and that $y|\mathbf{x},\mathbf{w}\sim\mathcal{N}(\mathbf{w}^T\mathbf{x},\sigma^2)$. Suppose as well that the data generating processes for $\mathbf{x},\mathbf{w},y$ are all independent (note that $\mathbf{w}$ is random!). Between the models with estimates $\mathbf{w}_{\mathrm{MAP}}$ and $\mathbf{w}_{\mathrm{mean}}$, which model would have a lower expected test MSE, and why? Assume that the data generating distribution for $\mathbf{x}$ has mean zero, and that distinct features are independent and each have variance 1.\footnote{The unit variance assumption simplifies computation, and is also commonly used in practical applications.}
        
    \end{enumerate}
  
  
\end{problem}

\subsection*{Solution:}
2.1 \begin{align*}
    p(\mathbf{w}|\mathbf{X}, \mathbf{y}) &\propto  p(\mathbf{y}|\mathbf{X},\mathbf{w}) p(\mathbf{w})\\
    &= \frac{1}{2\tau}\exp\left(-\frac{\|\mathbf{w}-\mathbf{\mu}\|_1}{\tau}\right) \prod_{i=1}^N \mathcal{N}(y_i|\bold{x}_i\bold{w}_i, \sigma^2) \\
    &\propto \exp\left(-\frac{\|\mathbf{w}-\mathbf{\mu}\|_1}{\tau}\right) \prod_{i=1}^N \exp \left(-\frac{(y_i - \bold{x}_i \bold{w}_1)^2}{2\sigma^2} \right)
\end{align*}

2.2 We find the MAP estimate by finding the $\bold{w}$ that maximizes the posterior. Therefore, 
\begin{align*}
    \bold{w}_{MAP} &= \argmax_{\bold{w}} p(\bold{w}|\bold{X}, \bold{y}) \\
    &= \argmax_{\bold{w}} \{ \exp\left(-\frac{\|\mathbf{w}-\mathbf{\mu}\|_1}{\tau}\right) \prod_{i=1}^N \exp \left(-\frac{(y_i - \bold{x}_i \bold{w}_1)^2}{2\sigma^2} \right) \}\\
    &= \argmax_{\bold{w}} \{-\frac{\|\mathbf{w}-\mathbf{\mu}\|_1}{\tau} + \sum_{i=1}^N -\frac{(y_i - \bold{x}_i \bold{w})^2}{2\sigma^2} \}\\
    &= \argmin{\bold{w}} \{ \frac{\|\mathbf{w}-\mathbf{\mu}\|_1}{\tau/2\sigma^2} + \sum_{i=1}^N (y_i - \bold{x}_i\bold{w})^2\}
\end{align*}

This is related to regularization in the frequentist perspective because this equation looks like minimizing the Lasso L1 regularization objective function for regression. The L1 regularization term corresponds to the $\frac{\|\mathbf{w}-\mathbf{\mu}\|_1}{\tau/2\sigma^2}$ term and the $\sum_{i=1}^N (y_i - \bold{x}_i\bold{w})^2$ term corresponds to the MSE.

The scale factor $\tau$ relates to the regularization parameter $\lambda$ because $\lambda = \frac{2\sigma^2}{\tau}$. Therefore, $\lambda \propto \frac{1}{\tau}$, so larger values of $\tau$ correspond with weaker regularization parameters and smaller $\tau$ corresponds to stronger regularization parameters $\lambda$. Intuitively, we hold a prior $\mu$ that we think weights should be similar to, so this regularization is penalizing weights that are very far to encourage $\bold{w}$ to be closer to $\mu$.

2.3 Prior expert knowledge can be used to construct a better prior distribution. If, for example, we knew beforehand that $\bold{w}$ should be close to a vector $\bold{v}$ in value, then we could set $\mu = \bold{v}$ as our prior. The MAP estimate of $\bold{w}$ is encouraged to be closer to $\mu$, so setting an appropriate prior would result in a more accurate resulting $\bold{w}$. From a Bayesian viewpoint, this expert knowledge positively influences our prior, and then the expert knowledge in the prior is then used to produce the posterior. From the frequentist viewpoint, the prior $p(\bold{w}) =\frac{1}{2\tau} \exp{( \frac{\|\mathbf{w}-\mathbf{v}\|_1}{\tau})}$ is greater if $\bold{w}$ is closer to $\bold{v}$, so we increase the likelihood that $\bold{w}$ is similar to $\bold{v}$.

2.4  As shown in 2.2, $\lambda \propto \frac{1}{\tau}$, so smaller $\tau$ correspond to weaker regularization parameters. As $\tau$ decreases, the entries in $\bold{w}_{MAP}$ get closer to $\mu = \bold{v}$ since regularization becomes stronger. As $\tau$ approaches 0, the limit approaches:
$$\lim_{\tau \rightarrow 0} = \argmin_{\bold{w}}  \frac{\|\mathbf{w}-\mathbf{\mu}\|_1}{\tau/2\sigma^2} + \sum_{i=1}^N (y_i - \bold{x}_i\bold{w})^2 = \argmin_{\bold{w}} {\| \bold{w} - \mu\|_1} = \mu$$




\newpage
% FDV: We had a lot of debate about this problem, whether it's useful
% for students to have to derive this by hand or not... if we want
% alternatives, ideas could include
%% (1) demonstrate the fact that there are good solutions near any
% starting point (run restarts from many starts, see that the answers
% have good quality but very different weights)
% (2) do a double-descent with linear regression (using very small N).

\begin{problem}[Neural Net Optimization]

  In this problem, we will take a closer look at how gradients are calculated for backprop with a simple multi-layer perceptron (MLP). The MLP will consist of a first fully connected layer with a sigmoid activation, followed by a one-dimensional, second fully connected layer with a sigmoid activation to get a prediction for a binary classification problem. Assume bias has not been merged. Let:
  \begin{itemize}
      \item $\bold{W}_1$ be the weights of the first layer, $\bold{b}_1$ be the bias of the first layer.
      \item $\bold{W}_2$ be the weights of the second layer, $\bold{b}_2$ be the bias of the second layer.
  \end{itemize}
  
  The described architecture can be written mathematically as: $$\hat{y} = \sigma(\bold{W}_2 \left[\sigma \left(\bold{W}_1 \bold{x} + \bold{b}_1\right)\right] + \bold{b}_2)$$
  
  where $\hat{y}$ is a scalar output of the net when passing in the single datapoint $\bold{x}$ (represented as a column vector), the additions are element-wise additions, and the sigmoid is an element-wise sigmoid.
  
  \begin{enumerate}
      \item Let:
      \begin{itemize}
          \item $N$ be the number of datapoints we have
          \item $M$ be the dimensionality of the data
          \item $H$ be the size of the hidden dimension of the first layer. Here, hidden dimension is used to describe the dimension of the resulting value after going through the layer. Based on the problem description, the hidden dimension of the second layer is 1.
      \end{itemize}
      
      Write out the dimensionality of each of the parameters, and of the intermediate variables:

          \begin{align*}
          \bold{a}_1 &= \bold{W}_1 \bold{x} + \bold{b}_1, 
          &\bold{z}_1 = \sigma(\bold{a}_1) \\
          a_2 &= \bold{W}_2 \bold{z}_1 + \bold{b}_2, 
          &\hat{y} = z_2 = \sigma(a_2)
          \end{align*}
          
      and make sure they work with the mathematical operations described above.
      
    \item  We will derive the gradients for each of the parameters.  The gradients can be used in gradient descent to find weights that improve our model's performance. For this question, assume there is only one datapoint $\bold{x}$, and that our loss is $L = -(y \log (\hat{y}) + (1 - y) \log (1 - \hat{y}))$. For all questions, the chain rule will be useful.
    \begin{enumerate}
        \item Find $\frac{\partial L}{\partial b_2}$. 
        
        \item Find $\frac{\partial L}{\partial W_2^h}$, where $W_2^h$ represents the $h$th element of $\bold{W}_2$.
        
        \item Find $\frac{\partial L}{\partial b_1^h}$, where $b_1^h$ represents the $h$th element of $\bold{b}_1$. (*Hint: Note that only the $h$th element of $\bold{a}_1$ and $\bold{z}_1$ depend on $b_1^h$ - this should help you with how to use the chain rule.)
        
        \item Find $\frac{\partial L}{\partial W_1^{h,m}}$, where  $W_1^{h,m}$ represents the element in row $h$, column $m$ in $\bold{W}_1$.
    
    \end{enumerate}
    \end{enumerate}
    
    \end{problem}

\subsection*{Solution:}
3.1 $\bold{W}_1$ takes in the weights of the first layer which has dimensions $M$ by 1, and $\bold{W}_1 \bold{x}$ outputs a vector that is $H$ by 1. Therefore, $\bold{W}_1$ must have dimension $H$ by $M$. $\bold{b}_1$ is added to a vector that is $H$ by 1, so it must also be of dimension $H$ by 1. The intermediate vector $\bold{a}_1$ should therefore be a $H$ by 1 vector, and applying the sigmoid function yields that $\bold{z}_1$ is also a $H$ by 1 vector. $\bold{W}_2$ takes in the weights of the second layer, which are $H$ by 1, and $\bold{W}_2 \bold{z}_1$ outputs a scalar value, so therefore $\bold{W}_2$ must be of dimension 1 by $H$. $\bold{b}_2$ is added to a scalar, so it must also be a 1 by 1 scalar. $a_2$ and $\hat{y}$ are therefore scalar values.

3.2 (a) The loss is:
\begin{align*}
    L &= -(y \log (\hat{y}) + (1 - y) \log (1 - \hat{y}))\\
    L &= -(y \log (\sigma(\bold{W}_2 \left[\sigma \left(\bold{a}_1\right)\right] + \bold{b}_2)) + (1 - y) \log (1 - \sigma(\bold{W}_2 \left[\sigma \left(\bold{a}_1\right)\right] + \bold{b}_2)))
    \intertext{Note that the $\bold{a}_1$ term is constant w.r.t $\bold{b}_2$, and we are trying to take the partial with regards to $\bold{b}_2$. Next, we take the derivative.}
    \frac{\partial L}{\partial b_2} &= -(y * \frac{(\sigma(\bold{W}_2 \left[\sigma \left(\bold{a}_1\right)\right] + \bold{b}_2)) (1-\sigma(\bold{W}_2 \left[\sigma \left(\bold{a}_1\right)\right] + \bold{b}_2))}{\sigma(\bold{W}_2 \left[\sigma \left(\bold{a}_1\right)\right] + \bold{b}_2)} - (1-y)* \frac{(1-\sigma(\bold{W}_2 \left[\sigma \left(\bold{a}_1\right)\right] + \bold{b}_2))(\sigma(\bold{W}_2 \left[\sigma \left(\bold{a}_1\right)\right] + \bold{b}_2))}{1-\sigma(\bold{W}_2 \left[\sigma \left(\bold{a}_1\right)\right] + \bold{b}_2)})\\
    &= -\Big(y (1-\sigma(\bold{W}_2 \left[\sigma \left(\bold{a}_1\right)\right] + \bold{b}_2)) - (1-y)(\sigma(\bold{W}_2 \left[\sigma \left(\bold{a}_1\right)\right] + \bold{b}_2))\Big)\\
    &= -y(1-\hat{y}) + (1-y) (\hat{y})\\
    &= \hat{y} - y
\end{align*}

3.2 (b) Using the chain rule, we have that 
\begin{align*}
    \frac{\partial L}{\partial W_2^h} &= \frac{\partial L}{\partial a_2} \frac{\partial a_2}{\partial W_2^h}\\
    \intertext{Note that $\frac{\partial L}{\partial a_2} = \frac{\partial L}{\partial b_2}$ since $a_2$ and $b_2$ have a linear relationship. Therefore, from part (a) we have that $\frac{\partial L}{\partial a_2} = \hat{y} - y$.}
    &= (\hat{y} - y) \frac{\partial a_2}{\partial W_2^h}\\
    &= (\hat{y} - y) (\bold{z}_1^h)
\end{align*}

3.3 (c) By the chain rule, we have that 
\begin{align*}
    \frac{\partial L}{\partial \bold{b}_1^h} &= \frac{\partial L}{\partial a_2} * \frac{\partial a_2}{\partial \bold{z}_1^h} * \frac{\partial \bold{z}_1^h}{\partial \bold{a}_1^h} * \frac{\partial \bold{a}_1^h}{\partial \bold{b}_1^h}\\
    &= (\hat{y} - y) (W_2^h) \bold{z}_1^h (1-\bold{z}_1^h)
\end{align*}
3.4 (d) Again using the chain rule, we have that
\begin{align*}
    \frac{\partial L}{\partial \bold{W}^{h,m}} &=\frac{\partial L}{\partial a_2} * \frac{\partial a_2}{\partial \bold{z}_1^h} * \frac{\partial \bold{z}_1^h}{\partial \bold{a}_1^h} * \frac{\partial \bold{a}_1^h}{\partial \bold{W}^{h,m}}\\
    &= (\hat{y} - y) (W_2^h) \bold{z}_1^h (1-\bold{z}_1^h) \bold{x}^m
\end{align*}



\newpage

\begin{problem}[Modern Deep Learning Tools: PyTorch]
  In this problem, you will learn how to use PyTorch. This machine learning library is massively popular and used heavily throughout industry and research. In \verb|T3_P3.ipynb| you will implement an MLP for image classification from scratch. Copy and paste code solutions below and include a final graph of your training progress. Also submit your completed \verb|T3_P3.ipynb| file.

  {\bfseries You will recieve no points for code not included below.}

  {\bfseries You will recieve no points for code using built-in APIs from the \verb|torch.nn| library.}
  
\end{problem}


\subsection*{Solution:}
Plot:

\includegraphics[width=\linewidth]{final_plot}

Code:

\begin{python}
n_inputs = 28 * 28
n_hiddens = 256
n_outputs = 10

W1 = torch.nn.Parameter(0.01*torch.randn(size=(n_inputs, n_hiddens)))
b1 = torch.nn.Parameter(torch.zeros(n_hiddens))
W2 = torch.nn.Parameter(0.01*torch.randn(size=(n_hiddens, n_outputs)))
b2 = torch.nn.Parameter(torch.zeros(n_outputs))



def relu(x):
    return torch.clamp(x, min = 0)



def softmax(X):
    z = torch.exp(X)
    return z/z.sum(1, keepdim = True)



def net(X):
  flattened_X = X.flatten(start_dim=1)
  H = relu(flattened_X@W1 + b1)
  O = softmax(H@W2 + b2)
  return O



def cross_entropy(y_hat, y):
  return -torch.log(y_hat[range(len(y_hat)), y])



def sgd(params, lr=0.1):
  with torch.no_grad():
        for i, param in enumerate(params):
            params[i] = params[i].sub_(lr*param.grad)
            params[i].grad.zero_()



def train(net, params, train_iter, loss_func=cross_entropy, updater=sgd):
  for _ in range(10):
        for X, y in train_iter:
            y_hat = net(X)
            avg_loss = loss_func(y_hat, y).mean()
            avg_loss.backward()
            updater(params)

\end{python}


\newpage
%%%%%%%%%%%%%%%%%%%%%%%%%%%%%%%%%%%%%%%%%%%%%
% Name and Calibration
%%%%%%%%%%%%%%%%%%%%%%%%%%%%%%%%%%%%%%%%%%%%%
\subsection*{Name}
Kathryn Zhou

\subsection*{Collaborators and Resources}
Whom did you work with, and did you use any resources beyond cs181-textbook and your notes?

Ethan Lee, Albert Zhang

\subsection*{Calibration}
Approximately how long did this homework take you to complete (in hours)? 

30


\end{document}

